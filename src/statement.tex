\documentclass[12pt,a4paper,oneside]{article}

\usepackage{cmap}
\usepackage[T2A]{fontenc}
\usepackage[utf8]{inputenc}
\usepackage[english,russian]{babel}
\usepackage[russian]{olymp}
\usepackage{graphicx}
\usepackage{amsmath,amssymb}
\usepackage{epigraph}
\usepackage{color}

\definecolor{coolgrey}{rgb}{0.55, 0.57, 0.67}
\definecolor{cream}{rgb}{1.0, 0.99, 0.82}
\definecolor{oldlavender}{rgb}{0.47, 0.41, 0.47}
\definecolor{pumpkin}{rgb}{1.0, 0.46, 0.09}
\definecolor{royalblue}{rgb}{0.25, 0.41, 0.88}
\definecolor{mikadoyellow}{rgb}{1.0, 0.77, 0.05}
\definecolor{limegreen}{rgb}{0.2, 0.8, 0.2}
\definecolor{melon}{rgb}{0.99, 0.74, 0.71}
\definecolor{orange-red}{rgb}{1.0, 0.27, 0.0}
\definecolor{pansypurple}{rgb}{0.47, 0.09, 0.29}
\definecolor{olive}{rgb}{0.5, 0.5, 0.0}
\definecolor{richelectricblue}{rgb}{0.03, 0.57, 0.82}
\definecolor{lava}{rgb}{0.81, 0.06, 0.13}
\definecolor{pinegreen}{rgb}{0.0, 0.47, 0.44}
\definecolor{goldenrod}{rgb}{0.85, 0.65, 0.13}
\definecolor{green-yellow}{rgb}{0.68, 1.0, 0.18}
\definecolor{indigo(web)}{rgb}{0.29, 0.0, 0.51}
\definecolor{harvestgold}{rgb}{0.85, 0.57, 0.0}


\usepackage[russian,colorlinks=true,urlcolor=red,linkcolor=oldlavender]{hyperref}
\usepackage{enumerate}
\usepackage{datetime}
\usepackage{lastpage}
\usepackage{import}
\usepackage{verbatim}

\input{contest_info.tex}

\renewcommand{\t}{\texttt}
\renewcommand{\le}{\leqslant}
\renewcommand{\ge}{\geqslant}

\binoppenalty=10000
\relpenalty=10000
\exhyphenpenalty=10000

\newcommand{\ProblemLabel}{undefined}
\newcommand{\ProblemTL}{undefined}
\newcommand{\ProblemML}{undefined}
\newcommand{\ProblemName}{undefined}

\newcommand{\q}[1]{\langle #1 \rangle}
\newcommand\NO[1]{\t{\##1}}
\def\O{\mathcal{O}}
\def\EPS{\varepsilon}
\def\SO{\Rightarrow}
\def\EQ{\Leftrightarrow}
\def\t{\texttt}
\def\XOR{\text{ {\raisebox{-2pt}{\ensuremath{\Hat{}}}} }}
\def\LINE{\vspace*{-1em}\noindent \underline{\hbox to 1\textwidth{{ } \hfil{ } \hfil{ } }}}

\def\probl#1#2#3#4{
  \renewcommand{\ProblemName}{#1}
  \renewcommand{\ProblemLabel}{#2}
  \renewcommand{\ProblemTL}{#3}
  \renewcommand{\ProblemML}{#4}
  \input ./problems/#2/#1.tex
}

\newcommand{\Section}[2]{
  \hbox{\hspace{1em}}
  \vspace*{-2.5em}
  \section*{\color{#1}{#2}}
  \addcontentsline{toc}{section}{\color{#1}{#2}}
  \vspace*{-0.5em}
}

\def\myindent{\hspace*{\parindent}\unskip}

\contest
{\NAME}%
{\WHERE}%
{\DATE}%

%\sectionfont{\fontsize{8}{8}\selectfont}

\def\compact{
  \setlength{\parskip}{0pt}
  \setlength{\itemsep}{0pt}
}

\begin{document}

\vspace*{-2.8em}

\tableofcontents

\vspace*{0.8em}
\LINE
\vspace*{0.8em}

%\pagebreak

\noindent{}Вы не умеете читать/выводить данные, открывать файлы? Воспользуйтесь
\href{http://acm.math.spbu.ru/~sk1/algo/sum/}{примерами}.
\vspace{0.6em}

%\url{http://acm.math.spbu.ru/~sk1/algo/input-output/io_export.cpp.html} \\
\noindent{}В некоторых задачах большой ввод и вывод.
Пользуйтесь \href{http://acm.math.spbu.ru/~sk1/algo/input-output/fread\_write\_export.cpp.html}{быстрым вводом-выводом}.

\vspace{0.6em}
\noindent{}\textbf{Обратите внимание}, что ввод-вывод во всех задачах стандартный.

\vspace{0.6em}
\noindent{}Задачи расположены в \textbf{произвольном порядке}!

%\href{http://acm.math.spbu.ru/~sk1/algo/memory.cpp.html}{переопределение стандартного аллокатора} ускорит вашу программу.

%\vspace{0.6em}
%\noindent{}Обратите внимание на компилятор \t{GNU C++11 5.1.0 (TDM-GCC-64) inc},
%который позволяет пользоваться \href{http://acm.math.spbu.ru/~sk1/algo/lib/optimization.h.html}{дополнительной библиотекой}.
%Под ним можно сдать \href{http://acm.math.spbu.ru/~sk1/algo/lib/}{вот это}.

\pagebreak

\input{problemset_info.tex}

\end{document}

